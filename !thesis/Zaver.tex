\chapter*{Závěr}

\addcontentsline{toc}{chapter}{Závěr}
\markboth{ZÁVĚR}{ZÁVĚR}
Prvním cílem této bakalářské práce je shrnutí základních teoretických poznatků týkajících se obecných NURBS ploch. Nejprve je v části \ref{cast21} zpracován matematický aparát -- afinní a projektivní prostory a zobrazení na nich -- potřebný při definici B-spline křivek a ploch a jejich racionálního rozšíření zvaného NURBS v částech \ref{cast22} a \ref{cast23}. Dále jsou popsány vlastnosti těchto křivek a ploch v části \ref{cast24}.

V části \ref{odvozeni} následuje odvození kružnicových oblouků jako NURBS křivek, kterého se využívá při tvorbě rotačních ploch. Jedná se o plochy, které vznikají rotací vstupní NURBS křivky kolem osy rotace. Pro ilustraci je v části \ref{revSurf} uvedeno několik příkladů rotačních NURBS ploch -- anuloid, kulová plocha.

Dalším cílem práce bylo programové zpracování problematiky rotačních NURBS ploch a jejich zobrazení. Pro splnění tohoto cíle jsem vytvořil aplikaci \texttt{NURBS.exe} (viz příloha). V~kapitole \ref{progCast} jsem zpracoval důležité algoritmy a postupy -- generování uzlového vektoru, De Boorův algoritmus a algoritmus pro vytváření rovnoběžkových kružnic. Dále je v části \ref{castProg} popsáno ovládání aplikace a nakonec jsou v části \ref{vystupy} zobrazeny výsledky, které byly touto aplikací získány.