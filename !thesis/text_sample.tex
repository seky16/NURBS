\chapter{První kapitola}
\label{Prvni}

\section{Singlet}
\label{singlet}

Řekněme, že potřebujeme singlet (jednoduchou čočku) F4 s~ohniskovou dálkou
100\,mm. F4 znamená, že průměr apertury čočky je $f/4=100/4=25$\,mm. Začneme
jednoduchým návrhem a~budeme jej postupně zdokonalovat.

V~editoru ZEMAXu   zadejte jednoduchou čočku $r_1=100$\,mm, tloušťka 4\,mm,
sklo BK7; $r_2=-100$\,mm. Povrch číslo 2 vložíte stisknutím klávesy Ins.
Postupujte podle obrrázku obr.~\ref{cocka}.
%
%
\begin{figure}[htb]
\begin{center}
   \includegraphics*[width=12cm,height=4cm,keepaspectratio]{obr/cockaedit}
\end{center}
\caption{Návrh čočky}
\label{cocka}
\end{figure}
%
%

\subsection{Základní údaje o~zobrazení}
\label{sing-zakludaje}

Zobrazte si návrh (obr.~\ref{sing-layout}).
%
%
\begin{figure}[htb]
\begin{center}
   \includegraphics*[width=8cm,height=8cm,keepaspectratio]{obr/layout}
\end{center}
\caption{Pohled na čočku}
\label{sing-layout}
\end{figure}
%
%
Tloušťka povrchu OBJ je Infinity což znamená, že předmět se nachází v~nekonečnu
před čočkou. Tloušťka povrchu 2 je 100 což znamená, že obrazová rovina IMG je
100\,mm za povrchem. Z~obrázku \ref{sing-layout} vidíme, že ohnisko je před
obrazovou rovinou. Efektivní ohnisková dálka této čočky je 97\,mm jak lze
zjistit v~\uv{status baru} programu ZEMAX.

Zakresleme si do grafu poněkud více paprsků. V~menu Settings můžeme nastavit
parametry grafu.


\subsubsection{Spot diagram}
\label{sing-spotsub}

Zobrazme si teď obraz našeho bodu v~nekonečnu před čočkou. Z~obrázku je vidět,
že průměr bodu je přibližně 2{,}0\,mm.
%
%

Ukázali jsme si několik základních typů grafů, které ukazují podobné údaje
různými cestami. Z~celkového pohledu jsme zjistili, že obraz se vytváří před
obrazovou rovinou kterou jsme zvolili. To nám potvrdil i~průběh podélné
otvorové vady, který nám současně řekl, že vzdálenost ohniska je přibližně
4\,mm.

Ze spot diagramu jsme zjistili, že bod v~nekonečnu se zobrazil do kroužku
o~průměru 2\,mm, stejný údaj jsme mohli získat také z~průběhu (příčné) otvorové
vady.

Ze sklonu průběhu otvorové vady v~počátku však můžeme vyčíst také to, že obraz
bodu je defokusován.

\subsection{Optimalizace zobrazení}
\label{sing-optimalizace}

Základní údaje o~čočce jsme již získali. Teď se pokusíme jich využít pro
získání co nejlepšího obrazu.

Pohlédneme-li na průběh otvorové vady pro oba předchozí způsoby určení polohy
obrazové roviny, mohlo by nás napadnout řešení optimalizace zobrazení. Viděli
jsme, že pro polohu obrazové roviny určenou z~paraxiálního paprsku byl sklon
průběhu otvorové vady v~počátku nulový a~vada neustále rostla. Ve druhém
případě byl sklon závislosti v~počátku záporný, vada nejprve klesala a~potom
rostla (pro určitý poloměr paprsku ve vstupní pupile byla dokonce nulová!). Co
kdybychom dali obrazovou rovinu někde mezi tyto dvě polohy. Obraz by byl sice
rozostřený, ale průměr kroužku by mohl být menší.

Optimální volba polohy obrazové roviny by byla v~nejužším místě svazku paprsků
(sedle).

Pohlédneme-li na průběh otvorové vady pro oba předchozí způsoby určení polohy
obrazové roviny, mohlo by nás napadnout řešení optimalizace zobrazení. Viděli
jsme, že pro polohu obrazové roviny určenou z~paraxiálního paprsku byl sklon
průběhu otvorové vady v~počátku nulový a~vada neustále rostla. Ve druhém
případě byl sklon závislosti v~počátku záporný, vada nejprve klesala a~potom
rostla (pro určitý poloměr paprsku ve vstupní pupile byla dokonce nulová!). Co
kdybychom dali obrazovou rovinu někde mezi tyto dvě polohy. Obraz by byl sice
rozostřený, ale průměr kroužku by mohl být menší.

Optimální volba polohy obrazové roviny by byla v~nejužším místě svazku paprsků
(sedle).

\section{Singlet2}
\label{singlet2}

Řekněme, že potřebujeme singlet (jednoduchou čočku) F4 s~ohniskovou dálkou
100\,mm. F4 znamená, že průměr apertury čočky je $f/4=100/4=25$\,mm. Začneme
jednoduchým návrhem a~budeme jej postupně zdokonalovat.

V~editoru ZEMAXu   zadejte jednoduchou čočku $r_1=100$\,mm, tloušťka 4\,mm,
sklo BK7; $r_2=-100$\,mm. Povrch číslo 2 vložíte stisknutím klávesy Ins.
Postupujte podle obrázku obr.~\ref{cocka}.
%
%
\begin{figure}[htb]
\begin{center}
   \includegraphics*[width=12cm,height=4cm,keepaspectratio]{obr/cockaedit}
\end{center}
\caption{Návrh čočky}
\label{cocka2}
\end{figure}
%
%

\subsection{Základní údaje o~zobrazení}
\label{sing-zakludaje2}

Zobrazte si návrh (obr.~\ref{sing-layout2}).
%
%
\begin{figure}[htb]
\begin{center}
   \includegraphics*[width=8cm,height=8cm,keepaspectratio]{obr/layout}
\end{center}
\caption{Pohled na čočku}
\label{sing-layout2}
\end{figure}
%
%
Tloušťka povrchu OBJ je Infinity což znamená, že předmět se nachází v~nekonečnu
před čočkou. Tloušťka povrchu 2 je 100 což znamená, že obrazová rovina IMG je
100\,mm za povrchem. Z~obrázku \ref{sing-layout2} vidíme, že ohnisko je před
obrazovou rovinou. Efektivní ohnisková dálka této čočky je 97\,mm jak lze
zjistit v~\uv{status baru} programu ZEMAX.

Zakresleme si do grafu poněkud více paprsků. V~menu Settings můžeme nastavit
parametry grafu.


\subsubsection{Spot diagram}
\label{sing-spotsub2}

Zobrazme si teď obraz našeho bodu v~nekonečnu před čočkou. Z~obrázku je vidět,
že průměr bodu je přibližně 2{,}0\,mm.
%
%

Ukázali jsme si několik základních typů grafů, které ukazují podobné údaje
různými cestami. Z~celkového pohledu jsme zjistili, že obraz se vytváří před
obrazovou rovinou kterou jsme zvolili. To nám potvrdil i~průběh podélné
otvorové vady, který nám současně řekl, že vzdálenost ohniska je přibližně
4\,mm.

Ze spot diagramu jsme zjistili, že bod v~nekonečnu se zobrazil do kroužku
o~průměru 2\,mm, stejný údaj jsme mohli získat také z~průběhu (příčné) otvorové
vady.

Ze sklonu průběhu otvorové vady v~počátku však můžeme vyčíst také to, že obraz
bodu je defokusován.

\subsection{Optimalizace zobrazení}
\label{sing-optimalizace2}

Základní údaje o~čočce jsme již získali. Teď se pokusíme jich využít pro
získání co nejlepšího obrazu.

Pohlédneme-li na průběh otvorové vady pro oba předchozí způsoby určení polohy
obrazové roviny, mohlo by nás napadnout řešení optimalizace zobrazení. Viděli
jsme, že pro polohu obrazové roviny určenou z~paraxiálního paprsku byl sklon
průběhu otvorové vady v~počátku nulový a~vada neustále rostla. Ve druhém
případě byl sklon závislosti v~počátku záporný, vada nejprve klesala a~potom
rostla (pro určitý poloměr paprsku ve vstupní pupile byla dokonce nulová!). Co
kdybychom dali obrazovou rovinu někde mezi tyto dvě polohy. Obraz by byl sice
rozostřený, ale průměr kroužku by mohl být menší.

Optimální volba polohy obrazové roviny by byla v~nejužším místě svazku paprsků
(sedle).
